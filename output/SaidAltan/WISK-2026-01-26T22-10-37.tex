\documentclass[letterpaper,11pt]{article}

\usepackage{latexsym}
\usepackage[empty]{fullpage}
\usepackage{titlesec}
\usepackage{marvosym}
\usepackage[usenames,dvipsnames]{color}
\usepackage{verbatim}
\usepackage{enumitem}
\usepackage[hidelinks]{hyperref}
\usepackage{fancyhdr}
\usepackage[english]{babel}
\usepackage{tabularx}
\input{glyphtounicode}


%----------FONT OPTIONS----------
% sans-serif
% \usepackage[sfdefault]{FiraSans}
% \usepackage[sfdefault]{roboto}
% \usepackage[sfdefault]{noto-sans}
% \usepackage[default]{sourcesanspro}

% serif
% \usepackage{CormorantGaramond}
% \usepackage{charter}


\pagestyle{fancy}
\fancyhf{} % clear all header and footer fields
\fancyfoot{}
\renewcommand{\headrulewidth}{0pt}
\renewcommand{\footrulewidth}{0pt}

% Adjust margins
\addtolength{\oddsidemargin}{-0.5in}
\addtolength{\evensidemargin}{-0.5in}
\addtolength{\textwidth}{1in}
\addtolength{\topmargin}{-.5in}
\addtolength{\textheight}{1.0in}

\urlstyle{same}

\raggedbottom
\raggedright
\setlength{\tabcolsep}{0in}

% Sections formatting
\titleformat{\section}{
  \vspace{-4pt}\scshape\raggedright\large
}{}{0em}{}[\color{black}\titlerule \vspace{-5pt}]

% Ensure that generate pdf is machine readable/ATS parsable
\pdfgentounicode=1

%-------------------------
% Custom commands
\newcommand{\resumeItem}[1]{
  \item\small{
    {#1 \vspace{-2pt}}
  }
}

\newcommand{\resumeSubheading}[4]{
  \vspace{-2pt}\item
    \begin{tabular*}{0.97\textwidth}[t]{l@{\extracolsep{\fill}}r}
      \textbf{#1} & #2 \\
      \textit{\small#3} & \textit{\small #4} \\
    \end{tabular*}\vspace{-7pt}
}

\newcommand{\resumeSubSubheading}[2]{
    \item
    \begin{tabular*}{0.97\textwidth}{l@{\extracolsep{\fill}}r}
      \textit{\small#1} & \textit{\small #2} \\
    \end{tabular*}\vspace{-7pt}
}

\newcommand{\resumeProjectHeading}[2]{
    \item
    \begin{tabular*}{0.97\textwidth}{l@{\extracolsep{\fill}}r}
      \small#1 & #2 \\
    \end{tabular*}\vspace{-7pt}
}

\newcommand{\resumeSubItem}[1]{\resumeItem{#1}\vspace{-4pt}}

\renewcommand\labelitemii{$\vcenter{\hbox{\tiny$\bullet$}}$}

\newcommand{\resumeSubHeadingListStart}{\begin{itemize}[leftmargin=0.15in, label={}]}
\newcommand{\resumeSubHeadingListEnd}{\end{itemize}}
\newcommand{\resumeItemListStart}{\begin{itemize}}
\newcommand{\resumeItemListEnd}{\end{itemize}\vspace{-5pt}}

%-------------------------------------------
%%%%%%  RESUME STARTS HERE  %%%%%%%%%%%%%%%%%%%%%%%%%%%%


\begin{document}

%----------HEADING----------
% \begin{tabular*}{\textwidth}{l@{\extracolsep{\fill}}r}
%   \textbf{\href{http://sourabhbajaj.com/}{\Large Sourabh Bajaj}} & Email : \href{mailto:sourabh@sourabhbajaj.com}{sourabh@sourabhbajaj.com}\\
%   \href{http://www.sourabhbajaj.com/}{http://www.sourabhbajaj.com} & Mobile : +1-123-456-7890 \\
% \end{tabular*}

\begin{center}
    \textbf{\Huge \scshape Said Altan} \\ \vspace{1pt}
    \small Software Developer $|$ {React \& Java} $|$ 
    {Building data-heavy, user-facing platforms} 
\end{center}

\begin{center}

    \small 343-988-3084 $|$ \href{mailto:x@x.com}{\underline{altansaid13@outlook.com}} $|$ 
    \href{https://linkedin.com/in/altansaid}{\underline{linkedin.com/in/altansaid}} $|$
    \href{https://github.com/altansaid}{\underline{github.com/altansaid}}
\end{center}


%-----------EDUCATION-----------
\section{Education}
  \resumeSubHeadingListStart
    \resumeSubheading
      {Algonquin College of Applied Arts and Technology}{Ottawa, ON}
      {Web Development and Internet Applications}{Sep 2023 -- April 2025}
    
  \resumeSubHeadingListEnd


%-----------EXPERIENCE-----------
\section{Experience}
  \resumeSubHeadingListStart

    \resumeSubheading
      {Software Developer}{2025-07 -- Present}
      {Arbitrage Cyclops}{Remote}
      \resumeItemListStart
        \resumeItem{Data-heavy B2B SaaS akışlarında production güvenilirliğini vurgulamak için pagination ve filtreleme davranışlarını \textbf{Go backend} servislerinin cursor-based yaklaşımıyla hizaladım; veri tutarlılığı ve öngörülebilir veri akışı sağladım.}
        \resumeItem{Büyük veri setlerinde (\textbf{250K+ kayıt}) kullanıcı akışlarını hızlandırmak için deterministik prefetching uyguladım; multi-second geçişleri near-instant navigasyona indirerek \textbf{performans} hedefini yükseltme yönünde somut iyileştirme sağladım.}
        \resumeItem{Cache doğrulamasını aktif filter state'e bağlayarak stale data reuse problemlerini iptal etmek üzere korumalar ekledim; \textbf{production-only} pagination/filtering tutarsızlıklarını \textbf{kalıcı} şekilde giderdim.}
        \resumeItem{Asenkron isteklerde \textbf{race condition}'ları azaltmak için stale response tespiti ve fetch/prefetch lifecycle guard'ları uyguladım; hata yönetimi ve \textbf{güvenilirlik} odağında daha stabil kullanıcı deneyimi sağladım.}
      \resumeItemListEnd

    \resumeSubheading
      {Web Developer Intern}{2024-09 -- 2024-12}
      {Solace}{Ottawa, Canada}
      \resumeItemListStart
        \resumeItem{Production web platformlarında \textbf{reusable bileşenler} geliştirerek davranış tutarlılığını artırdım; regresyon riskini azaltıp daha sürdürülebilir teslimat akışı sağladım.}
        \resumeItem{Kritik \textbf{production sorunlarını} analiz etme ve \textbf{kalıcı çözümler} üretme odağında, framework upgrade/runtime değişimleri ve deployment workflow kaynaklı problemleri ekip ile birlikte teşhis edip platform stabilitesini geri kazandırdım.}
        \resumeItem{Uzun-form, bilgi yoğun sayfalarda \textbf{state-driven UI} akışlarını yeniden tasarlayarak in-page navigation sistemini iyileştirdim; kullanılabilirlik ve içerik keşfedilebilirliğini artırdım.}
        \resumeItem{Lighthouse bulgularına göre \textbf{performans} iyileştirmeleri uygulayarak daha hızlı yükleme ve daha iyi standart uyumu sağladım; \textbf{kalite hedefini} yükseltme yaklaşımını sürdürdüm.}
      \resumeItemListEnd

  \resumeSubHeadingListEnd


%-----------PROJECTS-----------
\section{Projects}
    \resumeSubHeadingListStart
      \resumeProjectHeading{\href{https://plab2practice.com}{\textbf{PLAB2 Practice Platform}} $|$ \emph{Java, Spring Boot, PostgreSQL, Docker}}{\href{https://github.com/altansaid/plab2projectnew}{\underline{GitHub}}}
          \resumeItemListStart
            \resumeItem{Cloud-based systems üzerinde çalışan full-stack bir platformu uçtan uca geliştirdim; \textbf{Java + Spring Boot} ile \textbf{60+ REST API endpoint}'i olan backend servislerini katmanlı mimariyle kurguladım.}
            \resumeItem{Relational databases odağında \textbf{PostgreSQL} ile veri kalıcılığı sağladım; güvenilirlik ve \textbf{veri bütünlüğü} hedefleriyle kimlik doğrulama/rol bazlı erişimi JWT + Spring Security ile uyguladım.}
            \resumeItem{\textbf{Dockerize} servislerle production ortama \textbf{deploy} ederek canlı bir ürünü işlettim; 165 aktif kullanıcıya ulaşan platformda stabiliteyi koruyacak şekilde yayın sonrası bakım ve iyileştirmeleri sürdürdüm.}
          \resumeItemListEnd
      \resumeProjectHeading{\href{https://interviewcoach-ai.vercel.app/}{\textbf{Interview Coach AI}} $|$ \emph{Node.js, Express, MongoDB, Next.js}}{\href{https://github.com/altansaid/interviewcoach-ai}{\underline{GitHub}}}
          \resumeItemListStart
            \resumeItem{\textbf{Backend API geliştirme ve entegrasyon} odağında \textbf{Node.js + Express} ile uygulama servislerini geliştirdim; kullanıcı bağlamından kişiselleştirilmiş içerik üreten LLM API entegrasyonunu gerçekleştirdim.}
            \resumeItem{JWT ve Google OAuth ile kimlik doğrulama akışlarını uygulayarak \textbf{güvenli oturum yönetimi} sağladım; kullanıcı verilerinin korunmasına yönelik \textbf{temel güvenlik kontrollerini} yazdım.}
            \resumeItem{\textbf{MongoDB} üzerinde kalıcılık katmanını tasarlayarak kullanıcı oturumları/soru setleri/favoriler gibi verileri yönettim; dashboard üzerinden tekrar kullanılabilir çalışma akışını destekledim.}
          \resumeItemListEnd
    \resumeSubHeadingListEnd



%
%-----------PROGRAMMING SKILLS-----------
\section{Technical Skills}
 \begin{itemize}[leftmargin=0.15in, label={}]
    \small{\item{
     \textbf{Languages}{: Java, JavaScript, TypeScript, SQL, PHP, C\#} \\
     \textbf{Frameworks}{: Spring Boot, Spring Security, Node.js, Express, React, Next.js, Redux, Gatsby} \\
     \textbf{Developer Tools}{: Docker, Git, Firebase Authentication, Firestore, Redux Saga, Redux Thunk, Redux Persist, Jira} \\
     \textbf{Cloud \& Databases}{: PostgreSQL, MongoDB, Google Cloud Platform, Supabase, Render, Vercel}
    }}
 \end{itemize}


%-------------------------------------------
\end{document}