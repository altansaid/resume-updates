\documentclass[letterpaper,11pt]{article}

\usepackage{latexsym}
\usepackage[empty]{fullpage}
\usepackage{titlesec}
\usepackage{marvosym}
\usepackage[usenames,dvipsnames]{color}
\usepackage{verbatim}
\usepackage{enumitem}
\usepackage[hidelinks]{hyperref}
\usepackage{fancyhdr}
\usepackage[english]{babel}
\usepackage{tabularx}
\input{glyphtounicode}


%----------FONT OPTIONS----------
% sans-serif
% \usepackage[sfdefault]{FiraSans}
% \usepackage[sfdefault]{roboto}
% \usepackage[sfdefault]{noto-sans}
% \usepackage[default]{sourcesanspro}

% serif
% \usepackage{CormorantGaramond}
% \usepackage{charter}


\pagestyle{fancy}
\fancyhf{} % clear all header and footer fields
\fancyfoot{}
\renewcommand{\headrulewidth}{0pt}
\renewcommand{\footrulewidth}{0pt}

% Adjust margins
\addtolength{\oddsidemargin}{-0.5in}
\addtolength{\evensidemargin}{-0.5in}
\addtolength{\textwidth}{1in}
\addtolength{\topmargin}{-.5in}
\addtolength{\textheight}{1.0in}

\urlstyle{same}

\raggedbottom
\raggedright
\setlength{\tabcolsep}{0in}

% Sections formatting
\titleformat{\section}{
  \vspace{-4pt}\scshape\raggedright\large
}{}{0em}{}[\color{black}\titlerule \vspace{-5pt}]

% Ensure that generate pdf is machine readable/ATS parsable
\pdfgentounicode=1

%-------------------------
% Custom commands
\newcommand{\resumeItem}[1]{
  \item\small{
    {#1 \vspace{-2pt}}
  }
}

\newcommand{\resumeSubheading}[4]{
  \vspace{-2pt}\item
    \begin{tabular*}{0.97\textwidth}[t]{l@{\extracolsep{\fill}}r}
      \textbf{#1} & #2 \\
      \textit{\small#3} & \textit{\small #4} \\
    \end{tabular*}\vspace{-7pt}
}

\newcommand{\resumeSubSubheading}[2]{
    \item
    \begin{tabular*}{0.97\textwidth}{l@{\extracolsep{\fill}}r}
      \textit{\small#1} & \textit{\small #2} \\
    \end{tabular*}\vspace{-7pt}
}

\newcommand{\resumeProjectHeading}[2]{
    \item
    \begin{tabular*}{0.97\textwidth}{l@{\extracolsep{\fill}}r}
      \small#1 & #2 \\
    \end{tabular*}\vspace{-7pt}
}

\newcommand{\resumeSubItem}[1]{\resumeItem{#1}\vspace{-4pt}}

\renewcommand\labelitemii{$\vcenter{\hbox{\tiny$\bullet$}}$}

\newcommand{\resumeSubHeadingListStart}{\begin{itemize}[leftmargin=0.15in, label={}]}
\newcommand{\resumeSubHeadingListEnd}{\end{itemize}}
\newcommand{\resumeItemListStart}{\begin{itemize}}
\newcommand{\resumeItemListEnd}{\end{itemize}\vspace{-5pt}}

%-------------------------------------------
%%%%%%  RESUME STARTS HERE  %%%%%%%%%%%%%%%%%%%%%%%%%%%%


\begin{document}

%----------HEADING----------
% \begin{tabular*}{\textwidth}{l@{\extracolsep{\fill}}r}
%   \textbf{\href{http://sourabhbajaj.com/}{\Large Sourabh Bajaj}} & Email : \href{mailto:sourabh@sourabhbajaj.com}{sourabh@sourabhbajaj.com}\\
%   \href{http://www.sourabhbajaj.com/}{http://www.sourabhbajaj.com} & Mobile : +1-123-456-7890 \\
% \end{tabular*}

\begin{center}
    \textbf{\Huge \scshape Said Altan} \\ \vspace{1pt}
    \small Software Developer $|$ {React \& Java} $|$ 
    {Building data-heavy, user-facing platforms} 
\end{center}

\begin{center}

    \small 343-988-3084 $|$ \href{mailto:x@x.com}{\underline{altansaid13@outlook.com}} $|$ 
    \href{https://linkedin.com/in/altansaid}{\underline{linkedin.com/in/altansaid}} $|$
    \href{https://github.com/altansaid}{\underline{github.com/altansaid}}
\end{center}


%-----------EDUCATION-----------
\section{Education}
  \resumeSubHeadingListStart
    \resumeSubheading
      {Algonquin College of Applied Arts and Technology}{Ottawa, ON}
      {Web Development and Internet Applications}{Sep 2023 -- April 2025}
    
  \resumeSubHeadingListEnd


%-----------EXPERIENCE-----------
\section{Experience}
  \resumeSubHeadingListStart

    \resumeSubheading
      {Software Developer}{2025-07 -- Présent}
      {Arbitrage Cyclops}{Remote}
      \resumeItemListStart
        \resumeItem{Optimisé une stack web \textbf{React 18} (Redux/Redux Saga) pour une UI data-heavy en concevant un préchargement déterministe des pages adjacentes, réduisant des transitions de plusieurs secondes à une navigation quasi instantanée (\textbf{performance critique}).}
        \resumeItem{Amélioré la \textbf{qualité toujours en tête} des résultats (filtres/tri/pagination) en mettant en place une validation de cache liée à l'état des filtres, éliminant la réutilisation de données obsolètes observée en production.}
        \resumeItem{Renforcé la fiabilité des flux asynchrones en détectant/ignorant les réponses périmées (\textbf{Redux Sagas} + refs gardées), évitant des \textbf{race conditions} lors d'interactions rapides et stabilisant l'expérience utilisateur.}
        \resumeItem{Collaboré avec les responsables produit en \textbf{amélioration continue} pour investiguer des enjeux production et livrer des correctifs itératifs, alignés sur les besoins réels des utilisateurs.}
      \resumeItemListEnd

    \resumeSubheading
      {Web Developer Intern}{2024-09 -- 2024-12}
      {Solace}{Ottawa, Canada}
      \resumeItemListStart
        \resumeItem{Conçu et développé des composants front-end réutilisables (\textbf{React}, \textbf{TypeScript}) sur des plateformes web en production, réduisant les régressions via une logique d'interaction plus cohérente et maintenable.}
        \resumeItem{Optimisé une navigation in-page complexe en restructurant des flux UI pilotés par l'état pour des pages longues et denses, améliorant l'utilisabilité et la découvrabilité du contenu (\textbf{qualité} + \textbf{performance}).}
        \resumeItem{Amélioré performance, accessibilité et SEO en itérant sur des constats \textbf{Lighthouse} et bonnes pratiques, contribuant à des temps de chargement plus rapides et une meilleure conformité aux standards web.}
        \resumeItem{Collaboré en \textbf{méthodologie Agile} pour livrer des fonctionnalités prêtes production et résoudre des incidents liés à des upgrades de framework et workflows de déploiement, restaurant la stabilité de la plateforme.}
      \resumeItemListEnd

  \resumeSubHeadingListEnd


%-----------PROJECTS-----------
\section{Projects}
    \resumeSubHeadingListStart
      \resumeProjectHeading
          {\href{https://plab2practice.com}{\textbf{PLAB2 Practice Platform}} $|$ \emph{React, TypeScript, Spring Boot, PostgreSQL}}{\href{https://github.com/altansaid/plab2projectnew}{\underline{GitHub}}}
          \resumeItemListStart
            \resumeItem{Développé une plateforme full-stack temps réel (\textbf{React/TypeScript} + Spring Boot) permettant des sessions synchronisées via code partagé (timers, transitions d'état), pour simuler des consultations en conditions d'examen.}
            \resumeItem{Conçu un back-end modulaire avec \textbf{60+ endpoints} REST et authentification sécurisée (JWT, Spring Security), en gardant la \textbf{qualité toujours en tête} (séparation des rôles, extensibilité).}
            \resumeItem{Déployé en production via services conteneurisés (\textbf{Docker}) et assuré l'exploitation d'une plateforme active (\textbf{165 utilisateurs}), avec surveillance d'uptime et itérations guidées par retours utilisateurs.}
          \resumeItemListEnd
      \resumeProjectHeading
          {\href{https://interviewcoach-ai.vercel.app/}{\textbf{Interview Coach AI}} $|$ \emph{Next.js, React, Node.js, MongoDB}}{\href{https://github.com/altansaid/interviewcoach-ai}{\underline{GitHub}}}
          \resumeItemListStart
            \resumeItem{Conçu et développé une application full-stack (\textbf{stack web}) de préparation d'entrevue générant des questions personnalisées via API LLM, avec parcours structuré (rôle, contexte, objectifs) pour des résultats tangibles.}
            \resumeItem{Développé des APIs \textbf{Node.js/Express} avec persistance \textbf{MongoDB} pour gérer sessions, favoris et historique, en privilégiant des composants réutilisables côté UI et un modèle de données évolutif.}
            \resumeItem{Implémenté l'authentification sécurisée (\textbf{JWT}, Google OAuth) et un tableau de bord utilisateur pour organiser et reprendre les séries de questions, favorisant une expérience fluide et fiable.}
          \resumeItemListEnd
    \resumeSubHeadingListEnd



%
%-----------PROGRAMMING SKILLS-----------
\section{Technical Skills}
 \begin{itemize}[leftmargin=0.15in, label={}]
    \small{\item{
     \textbf{Languages}{: TypeScript, JavaScript, Java, SQL, PHP} \\
     \textbf{Frameworks}{: React, Node.js, Next.js, Express, Spring Boot, Gatsby, Redux} \\
     \textbf{Developer Tools}{: Docker, Git, Jira, Figma, Lighthouse, Google Analytics, Hotjar} \\
     \textbf{Databases}{: PostgreSQL, MongoDB, Firestore, Google Cloud Platform, Vercel, Render}
    }}
 \end{itemize}


%-------------------------------------------
\end{document}